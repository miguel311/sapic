\documentclass[xcolor=dvipsnames]{beamer}
<<<<<<< HEAD
\usetheme{Berkeley}
\usepackage[spanish]{babel}
\usepackage[utf8]{inputenc}
\usepackage[T1]{fontenc}
\usepackage{lmodern}
\usepackage{fancyhdr}
\usepackage{graphicx}
\usepackage{mathrsfs}
\usepackage{amsmath}
\usepackage{lmodern}
\usepackage{multimedia}
\usepackage{hyperref}
\usepackage{varioref}
\usepackage{ragged2e}
\usepackage{etoolbox}
\usepackage{lipsum}
\usepackage{fancyvrb}
\fvset{fontsize=\footnotesize}
\RecustomVerbatimEnvironment{verbatim}{Verbatim}{}
\apptocmd{\frame}{}{\justifying}{} % Allow optional arguments after frame.
\usecolortheme[{rgb={0.6,0,0}}]{structure}
\setbeamercovered{transparent}
\setbeamertemplate{items}[ball]
\setbeamertemplate{blocks}[rounded][shadow=true] 
\beamertemplateshadingbackground{gray!50}{white!50}
\newtheorem{Ejemplo}{Django}
\useoutertheme{infolines}
\title []{Framework Django 1.11}
\author{Instructor Ing. Erwin Paredes}
%\logo{ \includegraphics[height=3cm]{figura/logo-blanco2.png}}
\institute{CENDITEL}
\date{\today}

\begin{document}

\section{Cenditel}
\begin{frame}
\maketitle
\end{frame}

\subsection{La Fundación}
\begin{frame}
\begin{block}{Fundación CENDITEL}
\indent   
Somos el Centro Nacional de Desarrollo e Investigación en Tecnologías Libres, cuyas actividades son el desarrollo de proyectos con impacto tecnológico enmarcados en las áreas de Software, Hardware y Telecomunicaciones. 
https://www.cenditel.gob.ve

\end{block}
\end{frame}

\subsection{Finalidad del Curso}
\begin{frame}
\begin{block}{Finalidad}
     
Consolidar la comunidad de desarrollo en software libre alrededor de la tecnología, en este caso el framework de desarrollo Django del lenguaje de programación Python
\end{block}
\end{frame}

\subsection{Contenido del Curso}
\begin{frame}
\begin{block}{Contenido del Curso}
\begin{itemize}
\item 
Introducción
\item 
Configurar ambiente de trabajo
\item 
Crear proyecto
\item 
Crear aplicaciones y estructura del proyecto
\item 
Modelos y migraciones
\item 
Relaciones
\item 
Django shell y Querysets
\item 
Configurar URLs y primera views
\item 
Sistema de plantillas
\item 
Configurar archivos estáticos
\item 
Listar Registros
\item 
Formularios y vista para anexar registros
\item 
Vistas para modificar y eliminar registros
\end{itemize}
\end{block}
\end{frame}


\begin{frame}
\begin{block}{Contenido del Curso}
\begin{itemize}
\item
ListView, CreateView, UpdateView, DeleteView   
\item
Crear Login
\item
Registro de Usuarios
\item
Decorador Login required
\item
Recuperar contraseña por correo
\item
Introducción a conceptos avanzados
\end{itemize}
\end{block}
\end{frame}



\section{Django}
\subsection{Introducción}
\begin{frame}
Django: Es un entorno de trabajo para el desarrollo Web, basado en el lenguaje de programación Python.

Un entorno de trabajo o framework, implica una estructura conceptual y conjunto de herramientas que típicamente son un conjunto de librerías, que permite organizar el desarrollo.

Para el framework Django la estructura conceptual es el denominado MVC (Modelo Vista Controlador), esto es, se organiza el desarrollo en módulos que separan el modelo de datos, la interfaz o vista y el comportamiento.

Por lo que para cada aplicación o módulo que conforma el proyecto se agrupa en lo siguientes archivos:

\begin{itemize}
 \item models.py :Contiene las clases que definen la estructura de datos
 \item views.py  :Contiene la codificación de las funcionalidades que permiten controlar el comportamiento del sistema.
 \item *.html    :Contiene la codificación de la interfaz 
\end{itemize}
\end{frame}

\subsection{Configurar ambiente de trabajo}
\begin{frame}[fragile]

\begin{Large}\textbf{Sitio Oficial}\end{Large}
\begin{verbatim}

https://www.djangoproject.com/

Instalación

https://docs.djangoproject.com/en/1.11/intro/install

Descarga:

https://www.djangoproject.com/download/
\end{verbatim}
\end{frame}

\begin{frame}[fragile]

\begin{Large}\textbf{Instalación}\end{Large}
\begin{verbatim}
// Descomprimir el archivo

tar -xzvf Django1.11.tar.gz

cd Django1.11

python setup.py install

// Comprobar la instalación del framework:

python
import django
django.VERSION

Ctrl + D
\end{verbatim}
\end{frame}

\subsection{Crear proyecto}
\begin{frame}[fragile]

\begin{verbatim}
// Crear el primer proyecto:

django-admin startproject curso

// este comando crea la siguiente infraestructura
de carpetas y archivos:

curso/
    manage.py
    curso/
        __init__.py
        settings.py
        urls.py
        wsgi.py


Configurar la base de datos

\end{verbatim}
\end{frame}

\begin{frame}[fragile]

\begin{verbatim}
// Se debe haber instalado el gestor de la Base de
Datos Postgresql  y la librería de conexión psycopg2
// Creamos la base de datos:
su postgres
psql
CREATE USER administrador;
CREATE DATABASE curso;

// Se edita el archivo curso/settings.py la constante
DATABASES = {
   'default': {
        'ENGINE': 'django.db.backends.postgresql',
        'NAME': 'curso',
        'USER': 'administrador',
        'PASSWORD': 'clave',
        'HOST': '127.0.0.1',
        'PORT': '5432',
  } }
\end{verbatim}
\end{frame}

\begin{frame}[fragile]
\begin{verbatim}
// Se guarda y ejecuta el siguiente comando para
crear las tablas iniciales del proyecto:

python manage.py migrate

// Se ejcuta el comando para crear el superusuario o 
usuario administrador del proyecto:

python manage.py createsuperuser

// Se introducen los datos que solicita
// Se ejecuta el siguiente comando para activar el 
servidor de prueba de forma local

python manage.py runserver
\end{verbatim}
\end{frame}

\begin{frame}[fragile]
\begin{verbatim}
python manage.py runserver

// Arroja el siguiente mensaje
Performing system checks...

System check identified no issues (0 silenced).

July 26, 2017 - 15:32:30
Django version 1.11.3,using settings 'curso.settings'
Starting development server at http://127.0.0.1:8000/
Quit the server with CONTROL-C.

// Abrimos el navegador de preferencia y colocamos la
siguiente dirección para acceder al sistema
administrativo del proyecto:

http://localhost:8000/admin
\end{verbatim}
\end{frame}

\subsection{Crear aplicaciones}
\begin{frame}[fragile]
\begin{verbatim}

http://localhost:8000/admin

// En el debemos autenticarnos con el usuario y
password del superusuario creado en los pasos
anteriores

// Allí podemos gestionar los usuarios del sistema
así como los grupos que evenctualmente se utilizan 
para asignar permisos de acceso.

// En el terminal presionar las teclas Ctrl + C para
interrumpir la ejecución del servidor y así crear 
las aplicaciones del proyecto.

python manage.py startapp encuesta

\end{verbatim}
\end{frame}

\subsection{Modelos}
\begin{frame}[fragile]
\begin{verbatim}
python manage.py startapp encuesta

// Se edita el archivo encuesta/models.py

from django.db import models

class Question(models.Model):
    question_text = models.CharField(max_length=200)
    pub_date =models.DateTimeField('date published')

class Choice(models.Model):
    question = models.ForeignKey(Question, 
               on_delete=models.CASCADE)
    choice_text = models.CharField(max_length=200)
    votes = models.IntegerField(default=0)

\end{verbatim}
\end{frame}

\begin{frame}[fragile]
\begin{verbatim}
// Almacenamos y editamos el archivo 
curso/settings.py la constante:

INSTALLED_APPS = [
    ...
    'django.contrib.staticfiles',
    'encuesta',
]

// Para crear las tablas correspondiente de models.py
se ejecuta el comando:

python manage.py makemigrations encuesta

python manage.py migrate encuesta

\end{verbatim}
\end{frame}

\subsection{Relaciones}
\begin{frame}[fragile]
\begin{verbatim}
// Para el manejo de las tablas por medio del 
framework se puede realizar desde el terminal 
con el comando:

python manage.py shell


\end{verbatim}
\end{frame}

\subsection{Django shell y Querysets}
\begin{frame}[fragile]
\begin{verbatim}

// Allí podemos ejecutar las siguientes instrucciones

from encuesta.models import Question, Choice
from django.utils import timezone

Question.objects.all()
q = Question(question_text="Que hay de nuevo?", 
                     pub_date=timezone.now())
q.save()
q.id
q.question_text
q.pub_date
q.question_text = "Q ai d nuevo?"
q.save()

Ctril + D para salir
\end{verbatim}
\end{frame}

\begin{frame}[fragile]
\begin{verbatim}
Editamos nuevamente el archivo encuesta/models.py 
para añadir el sigueinte metodo a la clase Question

import datetime

from django.db import models
from django.utils import timezone

class Question(models.Model):
    # ...
    def was_published_recently(self):
        return self.pub_date >= timezone.now() -
                      datetime.timedelta(days=1)

Guardamos y volvemos a terminal:

python manage.py shell

\end{verbatim}
\end{frame}

\begin{frame}[fragile]
\begin{verbatim}
from polls.models import Question, Choice

Question.objects.all()
q = Question.objects.get(pk=1)
q.was_published_recently()

q = Question.objects.filter(id=1)

q = Question.objects.filter(
          question_text__startswith='Q')

from django.utils import timezone
current_year = timezone.now().year
Question.objects.get(pub_date__year=current_year)

Question.objects.get(id=2) Error...
\end{verbatim}
\end{frame}

\begin{frame}[fragile]
\begin{verbatim}
q = Question.objects.get(pk=1)
q.choice_set.all()
q.choice_set.create(choice_text='No mucho', votes=0)
q.choice_set.create(choice_text='Muchas cosas' 
                                           ,votes=0)
c = q.choice_set.create(choice_text='De todo un poco'
                                           ,votes=0)

c.question

q.choice_set.all()
q.choice_set.count()

Crtl + D para salir

\end{verbatim}
\end{frame}

\begin{frame}[fragile]
\begin{verbatim}
Incluir el modelo de la aplicación encuesta 
en el sistema admin:

Editamos el archivo encuesta/admin.py
from django.contrib import admin

from .models import Question

admin.site.register(Question)

Guardamos y ejecutamos el servidor 
de prueba nuevamente:

python manage.py runserver

en el navegador http://localhost:8000/admin

\end{verbatim}
\end{frame}

\subsection{Configurar URLs y primera views}
\begin{frame}[fragile]
\begin{verbatim}
Crear la primera vista para acceso del proyecto 
Editamos el archivo encuesta/urls.py

from django.conf.urls import url
from . import views
urlpatterns = [
    # ex: /encueta/
    url(r'^$', views.index, name='index'),
    # ex: /encuesta/5/
    url(r'^(?P<question_id>[0-9]+)/$', 
    views.detail, name='detail'),
    # ex: /encuesta/5/results/
    url(r'^(?P<question_id>[0-9]+)/results/$',
    views.results, name='results'),
    # ex: /encuesta/5/vote/
    url(r'^(?P<question_id>[0-9]+)/vote/$',
     views.vote, name='vote'),
]
\end{verbatim}
\end{frame}

\begin{frame}[fragile]
\begin{verbatim}
Editamos el archivo encuesta/views.py:

from django.http import HttpResponse
from .models import Question

def index(request):
    question_list = Question.objects.order_by('-pub_date')[:5]
    output = ', '.join([q.question_text for q in question_list])
    return HttpResponse(output)

def detail(request, question_id):
    return HttpResponse("Question No: %s." % question_id)

def results(request, question_id):
    response = "Results of question %s." 
    return HttpResponse(response % question_id)

def vote(request, question_id):
    return HttpResponse("Voting on question %s." % question_id)
\end{verbatim}
\end{frame}

\begin{frame}[fragile]
\begin{verbatim}
Guardamos y ejecutamos el servidor de prueba y 
probamos los enlaces desde el navegador:

python manage.py runserver

En el navegador probamos los siguientes enlaces:

http://localhost:8000/encueta
http://localhost:8000/encueta/1
http://localhost:8000/encueta/1/result
http://localhost:8000/encueta/1/vote

\end{verbatim}
\end{frame}

\subsection{Sistema de plantillas}

\begin{frame}[fragile]
\begin{verbatim}
Uso del plantillas:

Creamos la siguiente estructura

cd encuesta
mkdir templates
mkdir templates/encuesta

y editamos el archivo index.html allí

gedit encuesta/templates/encuesta/index.html

\end{verbatim}
\end{frame}

\begin{frame}[fragile]
\begin{verbatim}

    <ul>
    
        <li>
         <a href="/encuesta/{{ question.id }}/">
          {{
            question.question_text
          }}
         </a>
        </li>
    
    </ul>

    <p>No hay encuestas.</p>

\end{verbatim}
\end{frame}

\begin{frame}[fragile]
\begin{verbatim}
Guardamos y modificamos el archivo encuesta views.py 
para que utilice la plantilla

from django.http import HttpResponse
from django.template import loader

from .models import Question


def index(request):
    question_list = Question.objects.order_by('-pub_date')[:5]
    template = loader.get_template('polls/index.html')
    context = {
        'question_list': question_list,
    }
    return HttpResponse(template.render(context, request))

\end{verbatim}
\end{frame}

\begin{frame}[fragile]
\begin{verbatim}
o utilizando la shortcuts render

from django.shortcuts import render

from .models import Question


def index(request):
    question_list = Question.objects.order_by('-pub_date')[:5]
    context = {'question_list': question_list}
    return render(request, 'polls/index.html', context)


\end{verbatim}
\end{frame}

\subsection{Configurar estáticos}
\begin{frame}[fragile]
\begin{verbatim}
Descargar plantillas prediseñadas, por ejemplo:

https://adminlte.io/

Copiar los directorios ccs y js en los en la carpeta static

Copiar los archivo *.html en la carpeta templates

Editar index.html y cambiar los enlaces a cada archivo en:

<link rel="stylesheet" 
href="">

\end{verbatim}
\end{frame}

\subsection{Listar Registros}
\begin{frame}[fragile]
\begin{verbatim}
Creamos el archivo ajax.py

# -*- encoding: utf-8 -*-
from django.conf import settings
from django_datatables_view.base_datatable_view import (
   BaseDatatableView)
   
from django.contrib.auth.models import (
    User)

class ListUsersAjaxView(BaseDatatableView):
    model = User
    columns = ['pk','first_name','last_name','username','email',
               'date_joined', 'last_joined' ]
    order_columns = ['pk', 'username']
    max_display_length = 500
    
    def __init__(self):
        super(ListUsersAjaxView, self).__init__()

    def get_initial_queryset(self):
        return self.model.objects.all()
\end{verbatim}
\end{frame}

\begin{frame}[fragile]
\begin{verbatim}
def prepare_results(self, qs):
  json_data = []
  for item in qs:
    json_data.append([
      username,
     "{0} {1}".format(str(item.first_name),str(item.last_name)),
      item.email,
      item.date_joined.strftime("%Y-%m-%d %H:%M:%S"),
      last_login                
    ])
            
  return json_data
# Fin del archivo ajax.py

#En el archivo urls.py anexar las siguientes lineas
from .ajax import *

urlpatterns = [
    ....
url(r'^listar-users/$', ListUsersAjaxView.as_view(),
        name="listar_users"),
\end{verbatim}
\end{frame}

\begin{frame}[fragile]
\begin{verbatim}
Se anexa el siguiente código el archivo index.html
En la parte visual:

<div id="datatable"></div>

Y en la parte de código javascript

<script type="text/javascript">
$(document).ready(function() {
   $('#datatable').dataTable({
        "processing": true,
        "serverSide": true,
        "ajax": ,
        language: {url: JSON_DATA}
        });
    $('#datatable')
        .removeClass('display')
        .addClass('table table-striped table-bordered');
});
</script>
\end{verbatim}
\end{frame}

\subsection{Anexar registros}
\begin{frame}[fragile]
\begin{verbatim}
Creamos el método en el archivo views.py

def AnexarRegistro(request):
    if request.method == 'POST':
        
        vusername = request.POST['username']
        u = User(username = vusername)
        u.save()
        message = _("El usuario fue creado")
        template = loader.get_template('personal/profile.html')
    context = {'message':message}
    return HttpResponse(template.render(context, request))
       
\end{verbatim}
\end{frame}

\subsection{Modificar y eliminar registros}
\begin{frame}[fragile]
\begin{verbatim}
@login_required
def change_profile(request):
    """ Cambiar perfil de usuario
    """
    if request.method == 'POST':
        user_id = request.POST['user_id']
        personal = User.objects.get(user_id=request.user.id)               
        personal.first_name = request.POST['fname']
        personal.last_name = request.POST['lname']
        personal.email = request.POST['email']
        personal.is_active = request.POST.get('is_active')
        personal.user.save()
        
        message = _("The profile change was made")
	context = {'personal':personal,'message':message,}
    template = loader.get_template('personal/profile.html')
    return HttpResponse(template.render(context, request))
<<<<<<< HEAD
=======

\end{verbatim}
\end{frame}
>>>>>>> f837b9d91ddf1c492cd58e875f097fb99eb9339c

\begin{frame}[fragile]
\begin{verbatim}
@login_required
def delete_personal(request):
    """ Delete users 
    """
    id = json.loads(request.POST['seleccionados'])
    print id 
 
    for id_values in id:
        if id_values["id"] != "0":
	           
            u = User.objects.get(pk=int(id_values["id"]))
            u.delete()   
	   
    return redirect('personal')
\end{verbatim}
\end{frame}

\begin{frame}[fragile]
\begin{verbatim}
@login_required
def delete_personal(request):
    """ Delete users 
    """
    id = json.loads(request.POST['seleccionados'])
    print id 
 
    for id_values in id:
        if id_values["id"] != "0":
	           
            u = User.objects.get(pk=int(id_values["id"]))
            u.delete()   
	   
    return redirect('personal')
\end{verbatim}
\end{frame}

\subsection{View's}
\begin{frame}[fragile]
\begin{verbatim}
####  ListView

# En el archivo views.py

from django.views.generic.list import ListView
from django.utils import timezone

from articles.models import Article

class ArticleListView(ListView):

    model = Article

    def get_context_data(self, **kwargs):
        context = super(ArticleListView, 
                        self).get_context_data(**kwargs)
        context['now'] = timezone.now()
        return context

\end{verbatim}
\end{frame}

\begin{frame}[fragile]
\begin{verbatim}

# En el archivo urls.py
from django.conf.urls import url

from article.views import ArticleListView

urlpatterns = [
 url(r'^$',ArticleListView.as_view(),name='article-list'),
]

# El template
<h1>Articles</h1>
<ul>

    <li>
    {{ article.pub_date|date }} - {{ article.headline }}
    </li>

    <li>No articles yet.</li>

</ul>
\end{verbatim}
\end{frame}

\begin{frame}[fragile]
\begin{verbatim}
#### DetailView

# En el archivo views.py

from django.views.generic.detail import DetailView
from django.utils import timezone

from articles.models import Article

class ArticleDetailView(DetailView):

    model = Article

    def get_context_data(self, **kwargs):
        context = super(ArticleDetailView, self).
                  get_context_data(**kwargs)
        context['now'] = timezone.now()
        return context
        
\end{verbatim}
\end{frame}

\begin{frame}[fragile]
\begin{verbatim}

# En el archivo urls.py

from django.conf.urls import url

from article.views import ArticleDetailView

urlpatterns = [
 url(r'^(?P<slug>[-\w]+)/$',ArticleDetailView.as_view(),
                            name='article-detail'),
]

# El template

<h1>{{ object.headline }}</h1>
<p>{{ object.content }}</p>
<p>Reporter: {{ object.reporter }}</p>
<p>Published: {{ object.pub_date|date }}</p>
<p>Date: {{ now|date }}</p>
\end{verbatim}
\end{frame}

\begin{frame}[fragile]
\begin{verbatim}
#### CreateView

from django.views.generic.edit import CreateView
from myapp.models import Author

class AuthorCreate(CreateView):
    model = Author
    fields = ['name']

# En el template

<form action="" method="post">
    {{ form.as_p }}
    <input type="submit" value="Save" />
</form>

\end{verbatim}
\end{frame}

\begin{frame}[fragile]
\begin{verbatim}
#### UpdateView

from django.views.generic.edit import UpdateView
from myapp.models import Author

class AuthorUpdate(UpdateView):
    model = Author
    fields = ['name']
    template_name_suffix = '_update_form'
    
# En el template

<form action="" method="post">
    {{ form.as_p }}
    <input type="submit" value="Update" />
</form>

\end{verbatim}
\end{frame}

\begin{frame}[fragile]
\begin{verbatim}
#### DeleteView

from django.views.generic.edit import DeleteView
from django.urls import reverse_lazy
from myapp.models import Author

class AuthorDelete(DeleteView):
    model = Author
    success_url = reverse_lazy('author-list')
    
# En el template

<form action="" method="post">
    <p>Are you sure you want to delete "{{ object }}"?</p>
    <input type="submit" value="Confirm" />
</form>

\end{verbatim}
\end{frame}

\subsection{Crear Login}
\begin{frame}[fragile]
\begin{verbatim}
Creamos la aplicacion user desde la consola:

python manage.py startapp user

Editamos el archivo user/views.py:

# -*- coding: utf-8 -*-
<<<<<<< HEAD

from django.shortcuts import render
from django.contrib import messages
from django.contrib.auth import (
    authenticate, logout, login
)
from django.contrib.auth.models import (
    Group, Permission, User
)

=======

from django.shortcuts import render
from django.contrib import messages
from django.contrib.auth import (
    authenticate, logout, login
)
from django.contrib.auth.models import (
    Group, Permission, User
)

>>>>>>> f837b9d91ddf1c492cd58e875f097fb99eb9339c
class LoginView(FormView):
    form_class = FormularioLogin
    template_name = 'users.login.html'
    success_url = '/inicio/'
\end{verbatim}
\end{frame}

\begin{frame}[fragile]
\begin{verbatim}
def form_valid(self, form):
        
    usuario = form.cleaned_data['usuario']
    contrasena = form.cleaned_data['contrasena']
    usuario = authenticate(username=usuario,password=contrasena)
        
    if usuario is not None:
        login(self.request, usuario)
        messages.info(self.request,'Bienvenido %s has ingresado\
                                    Sistema con el usuario %s \
                                    ' % (usuario.first_name,
                                         usuario.username))
    else:
        self.success_url = reverse_lazy('users:login')
        messages.warning(self.request,'Verifique su nombre y \
                                       contraseña\
                                      y vuelve a intertar')

    return super(LoginView, self).form_valid(form)
 
\end{verbatim}
\end{frame}

\subsection{Registro de Usuarios}
\begin{frame}[fragile]
\begin{verbatim}
#### Crear Permisos desde el shell ejecutar

from myapp.models import BlogPost
from django.contrib.auth.models import Permission
from django.contrib.contenttypes.models import ContentType

content_type = ContentType.objects.get_for_model(BlogPost)
permission = Permission.objects.create(
    codename='can_publish',
    name='Can Publish Posts',
    content_type=content_type,
)

\end{verbatim}
\end{frame}

\begin{frame}[fragile]
\begin{verbatim}

from django.contrib.auth.models import Permission, User
from django.contrib.contenttypes.models import ContentType
from django.shortcuts import get_object_or_404

from myapp.models import BlogPost

def user_gains_perms(request, user_id):
    user = get_object_or_404(User, pk=user_id)
    # any permission check will 
    #cache the current set of permissions
    user.has_perm('myapp.change_blogpost')

    content_type=ContentType.objects.get_for_model(BlogPost)
    permission = Permission.objects.get(
        codename='change_blogpost',
        content_type=content_type,
    )
    user.user_permissions.add(permission)

\end{verbatim}
\end{frame}

\subsection{Login required}
\begin{frame}[fragile]
\begin{verbatim}

from django.contrib.auth.decorators import 
  login_required, permission_required
from django.views.generic import TemplateView

from .views import VoteView

urlpatterns = [
    url(r'^about/$', login_required(
           TemplateView.as_view(template_name="secret.html"))),
    url(r'^vote/$', permission_required(
           'polls.can_vote')(VoteView.as_view())),
]

\end{verbatim}
\end{frame}

\subsection{Recuperar contraseña por correo}
\begin{frame}[fragile]
\begin{verbatim}
def newpassword(request):
    puser = request.POST['puser']
    try:
        user = User.objects.get(username=puser)
        randompass = ''.join([choice(
        '1234567890qwertyuiopasdfghjklzxcvbnm') 
        for i in range(10)])
        print randompass
        subject = _('System: New Password')
        message = _('Your password is reset, new password: ')
                    + randompass
        user.email_user("subject","message")
        user.set_password(randompass)
        user.save()
    except:
        print "error send mail"
        mensaje = _("User not found")
    return redirect('/authenticate/signin')
 
\end{verbatim}
\end{frame}

\subsection{Conceptos avanzados}
\begin{frame}[fragile]
\begin{verbatim}
Otros conceptos

* Internacionalización
* Zonas Horarias
* Servidor Web
* Geodjango
* Django Rest

 Entre otros...
<<<<<<< HEAD
=======

\end{verbatim}
\end{frame}

\subsection{Geodjango}
\begin{frame}[fragile]
\begin{verbatim}
Para la gestión de datos geolocalizados debe utilizar 
una extensión de django denominada Geodjango

Instalación de componentes para la gestión de mapas
en particular la extensión django.contrib.gis

Entrar como root para la instalacion

aptitude install python3.4 python3-pip python3.4-dev 
python3-setuptools

aptitude install python3-virtualenv virtualenvwrapper

Salir del modo root y crear el ambiente:

$ mkvirtualenv --python=/usr/bin/python3 geodjango
\end{verbatim}
\end{frame}

\begin{frame}[fragile]
\begin{verbatim}
Instalar los requerimientos del proyecto

Para activar el ambiente virtual geodjango ejecute el comando:

$ workon geodjango

Quedando activado el entorno virtual de esta manera.

(geodjango)$

Crear el archivo requerimientos.txt
vim requerimientos.txt

*** Inicio del archivo requerimientos.txt
Django==1.11.4
psycopg2==2.7.1
pytz==2017.2
*** fin del archivo requeimientos.txt

Ejecutar el comando de descarga e instalación
pip install -r requeimientos.txt
\end{verbatim}
\end{frame}

\begin{frame}[fragile]
\begin{verbatim}
Crear el proyecto geodjango-demo

django-admin startproject geodjango-demo

cd geodjango-demo

Crear base de datos y Migrar los modelos
>>>>>>> f837b9d91ddf1c492cd58e875f097fb99eb9339c

Como super usuario instalar postgis:

aptitude install postgis

aptitude install postgresql-x.x-postgis-x.x

Nota: las x.x debe ser sustituidad por la versión de 
Postgres instalada
\end{verbatim}
\end{frame}

<<<<<<< HEAD
=======
\begin{frame}[fragile]
\begin{verbatim}
Ingresar a la consola de postgres con la siguiente orden:
$ psql

Ejecutar la siguiente sentencia estando en la consola de 
postgres:

postgres=# CREATE DATABASE geodjango OWNER=postgres;
postgres=#  \q

$ psql geodjango

geodjango=# CREATE EXTENSION postgis;
geodjango=# \q

Editar el archivo geodjango-demo/settings.py

INSTALLED_APPS = [
    ...
    'django.contrib.gis',
]
\end{verbatim}
\end{frame}

\begin{frame}[fragile]
\begin{verbatim}
Modificar la conexión para la base de datos:

DATABASES = {
    'default': {
        'ENGINE': 'django.contrib.gis.db.backends.postgis',
        'NAME': 'geodjango',
        'USER': 'administrador',
        'PASSWORD': 'clave',
        'HOST': 'localhost',
        'PORT': '5432',
        'ATOMIC_REQUESTS': True, 
        # Crea transacciones en cada peticion de la vista
    }
}

Para migrar los modelos del proyecto se debe usar 
el siguiente comando:

(geodjango)$ python manage.py makemigrations
(geodjango)$ python manage.py migrate
(geodjango)$ python manage.py createsuperusuer
\end{verbatim}
\end{frame}

\begin{frame}[fragile]
\begin{verbatim}
Creamos la aplicación geolocation
(geodjango)$ python manage.py startapp geolocation

Creamos el archivos geolocation/urls.py
# -*- coding: utf-8 -*-
from django.conf.urls import url
from .views import *
urlpatterns = [
                url(r'^register-poly/',
                    RegisterPolyView.as_view(),
                    name="register_poly"),
                url(r'^list-zipcode/',
                    ListZipcodeView.as_view(),
                    name="list_zipcode"),
                url(r'^delete-zipcode/(?P<pk>\d+)/',
                    ZipcodeDeleteView.as_view(),
                    name="delete_zipcode"),
                url(r'update-zipcode/(?P<pk>\d+)/$',
                    ZipCodeUpdate.as_view(),
                    name='update_zipcode')
              ]
\end{verbatim}
\end{frame}

\begin{frame}[fragile]
\begin{verbatim}
Modificar el archivo geolocation/views.py

# -*- coding: utf-8 -*-
from django.contrib import messages
from django.contrib.messages.views import SuccessMessageMixin
from django.urls import reverse_lazy
from django.views.generic.edit import (
    FormView, DeleteView, UpdateView
    )
from django.views.generic import ListView

from .forms import *
from .models import *

\end{verbatim}
\end{frame}

\begin{frame}[fragile]
\begin{verbatim}
class RegisterPolyView(FormView):
    form_class = ZipcodeForms
    template_name = 'geodjango-template.html'
    success_url = '/geolocation/register-poly/'
    def form_valid(self, form, **kwargs):
        new_zipcode = form.save()
        messages.success(self.request, 
        "ZipCode %s, registrado con exito" % (str(new_zipcode)))
        return super(RegisterPolyView, self).form_valid(form)
class ListZipcodeView(ListView):
    model = Zipcode
    template_name = 'geodjango-list.html'
    paginate_by = 3
class ZipcodeDeleteView(DeleteView):
    model = Zipcode
    success_url = reverse_lazy('geolocation:list_zipcode')
class ZipCodeUpdate(UpdateView, SuccessMessageMixin):
    model = Zipcode
    form_class = ZipcodeForms
    success_message = 'ZipCode Actualizado con exito'
    success_url = reverse_lazy('geolocation:list_zipcode')
\end{verbatim}
\end{frame}

\begin{frame}[fragile]
\begin{verbatim}
Editar el archivo admin.py
# -*- coding: utf-8 -*-
from django.contrib.gis import admin
from .models import *
admin.site.register(Zipcode, admin.OSMGeoAdmin)

Editar el archivo apps.py
from django.apps import AppConfig
class GeolocationConfig(AppConfig):
    name = 'geolocation'

Editar el archivo forms.py
# -*- coding: utf-8 -*-
from django.contrib.gis import forms
from .models import *

class ZipcodeForms(forms.ModelForm):
    class Meta:
        model = Zipcode
        fields = '__all__'
\end{verbatim}
\end{frame}

\begin{frame}[fragile]
\begin{verbatim}
    def __init__(self, *args, **kwargs):
        super(ZipcodeForms, self).__init__(*args, **kwargs)
        self.fields['code'].widget.attrs.update(
           {'class': 'form-control','placeholder': 'Zip Code'})
        self.fields['code'].label = 'Zip Code'
        self.fields['code'].required = True

        # Se le agrega la ruta donde se construye el mapa 
        # con el default_zoom
        self.fields['poly'].widget = 
         forms.OSMWidget.template_name = 'openlayers-cust.html'

        # Se le agrega al campo los atributos que por defecto 
        # tiene la ubicacion (lat lon) de Venezuela
        # Con un zoom por defecto de 5.2 y
        # Un alto y ancho de 600X400
        self.fields['poly'].widget = forms.OSMWidget(attrs={
                         'default_zoom': 5.2, 'map_width': 600,
                         'map_height': 400, 'default_lat': 8,
                         'default_lon': -66})
        self.fields['poly'].label = 'Cordenadas Poligonales'
        self.fields['poly'].required = True
\end{verbatim}
\end{frame}

\begin{frame}[fragile]
\begin{verbatim}
Editar el archivo models.py
# -*- coding: utf-8 -*-
from django.contrib.gis.db import models

class Zipcode(models.Model):
    code = models.CharField(max_length=5) #Campo del ZipCode
    poly = models.PolygonField() #Campo de la poligonal
    class Meta:
        ordering = ('code',)
        verbose_name = 'Zipcode'
        verbose_name_plural = 'Zipcodes'
    def __str__(self):
        return self.code
class Elevation(models.Model):
    name = models.CharField(max_length=100)
    rast = models.RasterField()
    class Meta:
        ordering = ('name',)
        verbose_name = 'Elevation'
        verbose_name_plural = 'Elevations'
    def __str__(self):
        return self.name
\end{verbatim}
\end{frame}

\begin{frame}[fragile]
\begin{verbatim}
Editar el archivo geodjango-demo/settings.py

INSTALLED_APPS = [
    ...
    'django.contrib.gis',
    'geolocation',
]

Guardamos y ejecutamos el comando 
(geodjango)$ python manage.py makemigrations geolocation
(geodjango)$ python manage.py migrate geolocation

Editamos el Templates para lo cual creamos la carpeta templates
y creamos los archivos geodjango-template.html y 
geodjango-list.html

mkdir geolocation/templates
cd geolocation/templates

\end{verbatim}
\end{frame}

\begin{frame}[fragile]
\begin{verbatim}
Archivo geodjango-list.html

Listar Zipcode 

    <h1 id="site-name"><a href="">
    Demo-GeoDjango</a></h1>


    <div class="breadcrumbs">
    <a href="">
    Registrar ZipCode</a>/
    <a href="">
    Listar ZipCode</a>
     &rsaquo; {{ title }}
    </div>


    <h1>Zip Codes</h1>
    <table><thead><tr><th>ZipCodes</th>
                      <th>Coordenadas</th>
                      <th>Acciones</th></tr></thead>
\end{verbatim}
\end{frame}

\begin{frame}[fragile]
\begin{verbatim}
    
        <tbody><tr>
         <td>{{ zips.code }}</td>
         <td>{{ zips.poly }}</td>
         <td><a title="Editar" class="changelink" 
          href="">
          </a>
         <a title="Eliminar" class="deletelink" 
          href="">
          </a>
         </td></tr></tbody>
    
        <tbody>
            <tr><td>No hay zipcode.</td></tr>
        </tbody>
    
 
    </table>
\end{verbatim}
\end{frame}

\begin{frame}[fragile]
\begin{verbatim}

    
        <div class="pagination">
            <span class="page-links">
                
<a href="/geolocation/list-zipcode/?page=
{{ page_obj.previous_page_number }}">Anterior</a>
                
                <span class="page-current">
Página {{ page_obj.number }}de{{ page_obj.paginator.num_pages }}
                </span>
                
<a href="/geolocation/list-zipcode/?page=
{{ page_obj.next_page_number }}">Siguiente</a>
                
          </span>
        </div>
    

""" Fin del archivo geodjango-list.html
\end{verbatim}
\end{frame}

\begin{frame}[fragile]
\begin{verbatim}
""" Archivo geodjango-templates.html

Registrar Zipcode 
{{ form.media }}
  <h1 id="site-name">
<a href="">Demo-GeoDjango</a></h1>


    <div class="breadcrumbs">
    <a href="">
    Registrar ZipCode</a>
    <a href="">
    Listar ZipCode</a>
     &rsaquo; {{ title }} </div>


    <form method="post">
        <div class="form-row">{{ form.as_p }}</div>
        <div class="submit-row">
         <input type="submit" value="Agregar ZipCode" />
        </div></form>

\end{verbatim}
\end{frame}

\begin{frame}[fragile]
\begin{verbatim}
""" Archivo openlayers-cust.html




{{ block.super }}
options['default_lon'] = {{ default_lon|unlocalize }};
options['default_lat'] = {{ default_lat|unlocalize }};
options['default_zoom'] = {{ default_zoom|unlocalize }};



var base_layer = new ol.layer.Tile({source: 
                                    new ol.source.OSM()});


\end{verbatim}
\end{frame}

\begin{frame}[fragile]
\begin{verbatim}


    <form method="post">
    
        <div class="form-row">
            {{ form.as_p }}
        </div>
        <div class="submit-row">
<label>&nbsp;</label>
<input type="submit" value="Agregar ZipCode" />
        </div>
    </form>


""" Fin del archivo geodjango-templates.html

\end{verbatim}
\end{frame}

\begin{frame}[fragile]
\begin{verbatim}
Editar el archivo geodjango-demo/urls.py

from django.conf.urls import url, include
from django.contrib.gis import admin

urlpatterns = [
    url(r'^admin/', admin.site.urls),
    url(r'^geolocation/', include('geolocation.urls', 
    namespace="geolocation")),
]

Guardar y ahora se puede ejecutar el comando 

(geodjango)$ python manage.py runserver

Y acceder desde un navegador la dirección 

http://localhost:8000/geolocation/register-poly/
\end{verbatim}
\end{frame}


>>>>>>> f837b9d91ddf1c492cd58e875f097fb99eb9339c
\begin{frame}[plain]

  
      \begin{center}

        \font\endfont = cmss10 at 15.40mm
        \color{Brown}
        \endfont 
        \baselineskip 20.0mm

        CENDITEL

      \end{center}    

    
\end{frame}


\end{document}
=======
2	\usetheme{Berkeley}
3	\usepackage[spanish]{babel}
4	\usepackage[utf8]{inputenc}
5	\usepackage[T1]{fontenc}
6	\usepackage{lmodern}
7	\usepackage{fancyhdr}
8	\usepackage{graphicx}
9	\usepackage{mathrsfs}
10	\usepackage{amsmath}
11	\usepackage{lmodern}
12	\usepackage{multimedia}
13	\usepackage{hyperref}
14	\usepackage{varioref}
15	\usepackage{ragged2e}
16	\usepackage{etoolbox}
17	\usepackage{lipsum}
18	\usepackage{fancyvrb}
19	\fvset{fontsize=\footnotesize}
20	\RecustomVerbatimEnvironment{verbatim}{Verbatim}{}
21	\apptocmd{\frame}{}{\justifying}{} % Allow optional arguments after frame.
22	\usecolortheme[{rgb={0.6,0,0}}]{structure}
23	\setbeamercovered{transparent}
24	\setbeamertemplate{items}[ball]
25	\setbeamertemplate{blocks}[rounded][shadow=true]
26	\beamertemplateshadingbackground{gray!50}{white!50}
27	\newtheorem{Ejemplo}{Django}
28	\useoutertheme{infolines}
29	\title []{Framework Django 1.11}
30	\author{Instructor Ing. Erwin Paredes}
31	%\logo{ \includegraphics[height=3cm]{figura/logo-blanco2.png}}
32	\institute{CENDITEL}
33	\date{\today}
34	
35	\begin{document}
36	
37	\section{Cenditel}
38	\begin{frame}
39	\maketitle
40	\end{frame}
41	
42	\subsection{La Fundación}
43	\begin{frame}
44	\begin{block}{Fundación CENDITEL}
45	\indent   
46	Somos el Centro Nacional de Desarrollo e Investigación en Tecnologías Libres, cuyas actividades son el desarrollo de proyectos con impacto tecnológico enmarcados en las áreas de Software, Hardware y Telecomunicaciones.
47	https://www.cenditel.gob.ve
48	
49	\end{block}
50	\end{frame}
51	
52	\subsection{Finalidad del Curso}
53	\begin{frame}
54	\begin{block}{Finalidad}
55	     
56	Consolidar la comunidad de desarrollo en software libre alrededor de la tecnología, en este caso el framework de desarrollo Django del lenguaje de programación Python
57	\end{block}
58	\end{frame}
59	
60	\subsection{Contenido del Curso}
61	\begin{frame}
62	\begin{block}{Contenido del Curso}
63	\begin{itemize}
64	\item 
65	Introducción
66	\item 
67	Configurar ambiente de trabajo
68	\item 
69	Crear proyecto
70	\item 
71	Crear aplicaciones y estructura del proyecto
72	\item 
73	Modelos y migraciones
74	\item 
75	Relaciones
76	\item 
77	Django shell y Querysets
78	\item 
79	Configurar URLs y primera views
80	\item 
81	Sistema de plantillas
82	\item 
83	Configurar archivos estáticos
84	\item 
85	Listar Registros
86	\item 
87	Formularios y vista para anexar registros
88	\item 
89	Vistas para modificar y eliminar registros
90	\end{itemize}
91	\end{block}
92	\end{frame}
93	
94	
95	\begin{frame}
96	\begin{block}{Contenido del Curso}
97	\begin{itemize}
98	\item
99	ListView, CreateView, UpdateView, DeleteView   
100	\item
101	Crear Login
102	\item
103	Registro de Usuarios
104	\item
105	Decorador Login required
106	\item
107	Recuperar contraseña por correo
108	\item
109	Introducción a conceptos avanzados
110	\end{itemize}
111	\end{block}
112	\end{frame}
113	
114	
115	
116	\section{Django}
117	\subsection{Introducción}
118	\begin{frame}
119	Django: Es un entorno de trabajo para el desarrollo Web, basado en el lenguaje de programación Python.
120	
121	Un entorno de trabajo o framework, implica una estructura conceptual y conjunto de herramientas que típicamente son un conjunto de librerías, que permite organizar el desarrollo.
122	
123	Para el framework Django la estructura conceptual es el denominado MVC (Modelo Vista Controlador), esto es, se organiza el desarrollo en módulos que separan el modelo de datos, la interfaz o vista y el comportamiento.
124	
125	Por lo que para cada aplicación o módulo que conforma el proyecto se agrupa en lo siguientes archivos:
126	
127	\begin{itemize}
128	 \item models.py :Contiene las clases que definen la estructura de datos
129	 \item views.py  :Contiene la codificación de las funcionalidades que permiten controlar el comportamiento del sistema.
130	 \item *.html    :Contiene la codificación de la interfaz
131	\end{itemize}
132	\end{frame}
133	
134	\subsection{Configurar ambiente de trabajo}
135	\begin{frame}[fragile]
136	
137	\begin{Large}\textbf{Sitio Oficial}\end{Large}
138	\begin{verbatim}
139	
140	https://www.djangoproject.com/
141	
142	Instalación
143	
144	https://docs.djangoproject.com/en/1.11/intro/install
145	
146	Descarga:
147	
148	https://www.djangoproject.com/download/
149	\end{verbatim}
150	\end{frame}
151	
152	\begin{frame}[fragile]
153	
154	\begin{Large}\textbf{Instalación}\end{Large}
155	\begin{verbatim}
156	// Descomprimir el archivo
157	
158	tar -xzvf Django1.11.tar.gz
159	
160	cd Django1.11
161	
162	python setup.py install
163	
164	// Comprobar la instalación del framework:
165	
166	python
167	import django
168	django.VERSION
169	
170	Ctrl + D
171	\end{verbatim}
172	\end{frame}
173	
174	\subsection{Crear proyecto}
175	\begin{frame}[fragile]
176	
177	\begin{verbatim}
178	// Crear el primer proyecto:
179	
180	django-admin startproject curso
181	
182	// este comando crea la siguiente infraestructura
183	de carpetas y archivos:
184	
185	curso/
186	    manage.py
187	    curso/
188	        __init__.py
189	        settings.py
190	        urls.py
191	        wsgi.py
192	
193	
194	Configurar la base de datos
195	
196	\end{verbatim}
197	\end{frame}
198	
199	\begin{frame}[fragile]
200	
201	\begin{verbatim}
202	// Se debe haber instalado el gestor de la Base de
203	Datos Postgresql  y la librería de conexión psycopg2
204	// Creamos la base de datos:
205	su postgres
206	psql
207	CREATE USER administrador;
208	CREATE DATABASE curso;
209	
210	// Se edita el archivo curso/settings.py la constante
211	DATABASES = {
212	   'default': {
213	        'ENGINE': 'django.db.backends.postgresql',
214	        'NAME': 'curso',
215	        'USER': 'administrador',
216	        'PASSWORD': 'clave',
217	        'HOST': '127.0.0.1',
218	        'PORT': '5432',
219	  } }
220	\end{verbatim}
221	\end{frame}
222	
223	\begin{frame}[fragile]
224	\begin{verbatim}
225	// Se guarda y ejecuta el siguiente comando para
226	crear las tablas iniciales del proyecto:
227	
228	python manage.py migrate
229	
230	// Se ejcuta el comando para crear el superusuario o
231	usuario administrador del proyecto:
232	
233	python manage.py createsuperuser
234	
235	// Se introducen los datos que solicita
236	// Se ejecuta el siguiente comando para activar el
237	servidor de prueba de forma local
238	
239	python manage.py runserver
240	\end{verbatim}
241	\end{frame}
242	
243	\begin{frame}[fragile]
244	\begin{verbatim}
245	python manage.py runserver
246	
247	// Arroja el siguiente mensaje
248	Performing system checks...
249	
250	System check identified no issues (0 silenced).
251	
252	July 26, 2017 - 15:32:30
253	Django version 1.11.3,using settings 'curso.settings'
254	Starting development server at http://127.0.0.1:8000/
255	Quit the server with CONTROL-C.
256	
257	// Abrimos el navegador de preferencia y colocamos la
258	siguiente dirección para acceder al sistema
259	administrativo del proyecto:
260	
261	http://localhost:8000/admin
262	\end{verbatim}
263	\end{frame}
264	
265	\subsection{Crear aplicaciones}
266	\begin{frame}[fragile]
267	\begin{verbatim}
268	
269	http://localhost:8000/admin
270	
271	// En el debemos autenticarnos con el usuario y
272	password del superusuario creado en los pasos
273	anteriores
274	
275	// Allí podemos gestionar los usuarios del sistema
276	así como los grupos que evenctualmente se utilizan
277	para asignar permisos de acceso.
278	
279	// En el terminal presionar las teclas Ctrl + C para
280	interrumpir la ejecución del servidor y así crear
281	las aplicaciones del proyecto.
282	
283	python manage.py startapp encuesta
284	
285	\end{verbatim}
286	\end{frame}
287	
288	\subsection{Modelos}
289	\begin{frame}[fragile]
290	\begin{verbatim}
291	python manage.py startapp encuesta
292	
293	// Se edita el archivo encuesta/models.py
294	
295	from django.db import models
296	
297	class Question(models.Model):
298	    question_text = models.CharField(max_length=200)
299	    pub_date =models.DateTimeField('date published')
300	
301	class Choice(models.Model):
302	    question = models.ForeignKey(Question,
303	               on_delete=models.CASCADE)
304	    choice_text = models.CharField(max_length=200)
305	    votes = models.IntegerField(default=0)
306	
307	\end{verbatim}
308	\end{frame}
309	
310	\begin{frame}[fragile]
311	\begin{verbatim}
312	// Almacenamos y editamos el archivo
313	curso/settings.py la constante:
314	
315	INSTALLED_APPS = [
316	    ...
317	    'django.contrib.staticfiles',
318	    'encuesta',
319	]
320	
321	// Para crear las tablas correspondiente de models.py
322	se ejecuta el comando:
323	
324	python manage.py makemigrations encuesta
325	
326	python manage.py migrate encuesta
327	
328	\end{verbatim}
329	\end{frame}
330	
331	\subsection{Relaciones}
332	\begin{frame}[fragile]
333	\begin{verbatim}
334	// Para el manejo de las tablas por medio del
335	framework se puede realizar desde el terminal
336	con el comando:
337	
338	python manage.py shell
339	
340	
341	\end{verbatim}
342	\end{frame}
343	
344	\subsection{Django shell y Querysets}
345	\begin{frame}[fragile]
346	\begin{verbatim}
347	
348	// Allí podemos ejecutar las siguientes instrucciones
349	
350	from encuesta.models import Question, Choice
351	from django.utils import timezone
352	
353	Question.objects.all()
354	q = Question(question_text="Que hay de nuevo?",
355	                     pub_date=timezone.now())
356	q.save()
357	q.id
358	q.question_text
359	q.pub_date
360	q.question_text = "Q ai d nuevo?"
361	q.save()
362	
363	Ctril + D para salir
364	\end{verbatim}
365	\end{frame}
366	
367	\begin{frame}[fragile]
368	\begin{verbatim}
369	Editamos nuevamente el archivo encuesta/models.py
370	para añadir el sigueinte metodo a la clase Question
371	
372	import datetime
373	
374	from django.db import models
375	from django.utils import timezone
376	
377	class Question(models.Model):
378	    # ...
379	    def was_published_recently(self):
380	        return self.pub_date >= timezone.now() -
381	                      datetime.timedelta(days=1)
382	
383	Guardamos y volvemos a terminal:
384	
385	python manage.py shell
386	
387	\end{verbatim}
388	\end{frame}
389	
390	\begin{frame}[fragile]
391	\begin{verbatim}
392	from polls.models import Question, Choice
393	
394	Question.objects.all()
395	q = Question.objects.get(pk=1)
396	q.was_published_recently()
397	
398	q = Question.objects.filter(id=1)
399	
400	q = Question.objects.filter(
401	          question_text__startswith='Q')
402	
403	from django.utils import timezone
404	current_year = timezone.now().year
405	Question.objects.get(pub_date__year=current_year)
406	
407	Question.objects.get(id=2) Error...
408	\end{verbatim}
409	\end{frame}
410	
411	\begin{frame}[fragile]
412	\begin{verbatim}
413	q = Question.objects.get(pk=1)
414	q.choice_set.all()
415	q.choice_set.create(choice_text='No mucho', votes=0)
416	q.choice_set.create(choice_text='Muchas cosas'
417	                                           ,votes=0)
418	c = q.choice_set.create(choice_text='De todo un poco'
419	                                           ,votes=0)
420	
421	c.question
422	
423	q.choice_set.all()
424	q.choice_set.count()
425	
426	Crtl + D para salir
427	
428	\end{verbatim}
429	\end{frame}
430	
431	\begin{frame}[fragile]
432	\begin{verbatim}
433	Incluir el modelo de la aplicación encuesta
434	en el sistema admin:
435	
436	Editamos el archivo encuesta/admin.py
437	from django.contrib import admin
438	
439	from .models import Question
440	
441	admin.site.register(Question)
442	
443	Guardamos y ejecutamos el servidor
444	de prueba nuevamente:
445	
446	python manage.py runserver
447	
448	en el navegador http://localhost:8000/admin
449	
450	\end{verbatim}
451	\end{frame}
452	
453	\subsection{Configurar URLs y primera views}
454	\begin{frame}[fragile]
455	\begin{verbatim}
456	Crear la primera vista para acceso del proyecto
457	Editamos el archivo encuesta/urls.py
458	
459	from django.conf.urls import url
460	from . import views
461	urlpatterns = [
462	    # ex: /encueta/
463	    url(r'^$', views.index, name='index'),
464	    # ex: /encuesta/5/
465	    url(r'^(?P<question_id>[0-9]+)/$',
466	    views.detail, name='detail'),
467	    # ex: /encuesta/5/results/
468	    url(r'^(?P<question_id>[0-9]+)/results/$',
469	    views.results, name='results'),
470	    # ex: /encuesta/5/vote/
471	    url(r'^(?P<question_id>[0-9]+)/vote/$',
472	     views.vote, name='vote'),
473	]
474	\end{verbatim}
475	\end{frame}
476	
477	\begin{frame}[fragile]
478	\begin{verbatim}
479	Editamos el archivo encuesta/views.py:
480	
481	from django.http import HttpResponse
482	from .models import Question
483	
484	def index(request):
485	    question_list = Question.objects.order_by('-pub_date')[:5]
486	    output = ', '.join([q.question_text for q in question_list])
487	    return HttpResponse(output)
488	
489	def detail(request, question_id):
490	    return HttpResponse("Question No: %s." % question_id)
491	
492	def results(request, question_id):
493	    response = "Results of question %s."
494	    return HttpResponse(response % question_id)
495	
496	def vote(request, question_id):
497	    return HttpResponse("Voting on question %s." % question_id)
498	\end{verbatim}
499	\end{frame}
500	
501	\begin{frame}[fragile]
502	\begin{verbatim}
503	Guardamos y ejecutamos el servidor de prueba y
504	probamos los enlaces desde el navegador:
505	
506	python manage.py runserver
507	
508	En el navegador probamos los siguientes enlaces:
509	
510	http://localhost:8000/encueta
511	http://localhost:8000/encueta/1
512	http://localhost:8000/encueta/1/result
513	http://localhost:8000/encueta/1/vote
514	
515	\end{verbatim}
516	\end{frame}
517	
518	\subsection{Sistema de plantillas}
519	
520	\begin{frame}[fragile]
521	\begin{verbatim}
522	Uso del plantillas:
523	
524	Creamos la siguiente estructura
525	
526	cd encuesta
527	mkdir templates
528	mkdir templates/encuesta
529	
530	y editamos el archivo index.html allí
531	
532	gedit encuesta/templates/encuesta/index.html
533	
534	\end{verbatim}
535	\end{frame}
536	
537	\begin{frame}[fragile]
538	\begin{verbatim}
539	
540	    <ul>
541	    
542	        <li>
543	         <a href="/encuesta/{{ question.id }}/">
544	          {{
545	            question.question_text
546	          }}
547	         </a>
548	        </li>
549	    
550	    </ul>
551	
552	    <p>No hay encuestas.</p>
553	
554	\end{verbatim}
555	\end{frame}
556	
557	\begin{frame}[fragile]
558	\begin{verbatim}
559	Guardamos y modificamos el archivo encuesta views.py
560	para que utilice la plantilla
561	
562	from django.http import HttpResponse
563	from django.template import loader
564	
565	from .models import Question
566	
567	
568	def index(request):
569	    question_list = Question.objects.order_by('-pub_date')[:5]
570	    template = loader.get_template('polls/index.html')
571	    context = {
572	        'question_list': question_list,
573	    }
574	    return HttpResponse(template.render(context, request))
575	
576	\end{verbatim}
577	\end{frame}
578	
579	\begin{frame}[fragile]
580	\begin{verbatim}
581	o utilizando la shortcuts render
582	
583	from django.shortcuts import render
584	
585	from .models import Question
586	
587	
588	def index(request):
589	    question_list = Question.objects.order_by('-pub_date')[:5]
590	    context = {'question_list': question_list}
591	    return render(request, 'polls/index.html', context)
592	
593	
594	\end{verbatim}
595	\end{frame}
596	
597	\subsection{Configurar estáticos}
598	\begin{frame}[fragile]
599	\begin{verbatim}
600	Descargar plantillas prediseñadas, por ejemplo:
601	
602	https://adminlte.io/
603	
604	Copiar los directorios ccs y js en los en la carpeta static
605	
606	Copiar los archivo *.html en la carpeta templates
607	
608	Editar index.html y cambiar los enlaces a cada archivo en:
609	
610	<link rel="stylesheet"
611	href="">
612	
613	\end{verbatim}
614	\end{frame}
615	
616	\subsection{Listar Registros}
617	\begin{frame}[fragile]
618	\begin{verbatim}
619	Creamos el archivo ajax.py
620	
621	# -*- encoding: utf-8 -*-
622	from django.conf import settings
623	from django_datatables_view.base_datatable_view import (
624	   BaseDatatableView)
625	   
626	from django.contrib.auth.models import (
627	    User)
628	
629	class ListUsersAjaxView(BaseDatatableView):
630	    model = User
631	    columns = ['pk','first_name','last_name','username','email',
632	               'date_joined', 'last_joined' ]
633	    order_columns = ['pk', 'username']
634	    max_display_length = 500
635	   
636	    def __init__(self):
637	        super(ListUsersAjaxView, self).__init__()
638	
639	    def get_initial_queryset(self):
640	        return self.model.objects.all()
641	\end{verbatim}
642	\end{frame}
643	
644	\begin{frame}[fragile]
645	\begin{verbatim}
646	def prepare_results(self, qs):
647	  json_data = []
648	  for item in qs:
649	    json_data.append([
650	      username,
651	     "{0} {1}".format(str(item.first_name),str(item.last_name)),
652	      item.email,
653	      item.date_joined.strftime("%Y-%m-%d %H:%M:%S"),
654	      last_login               
655	    ])
656	           
657	  return json_data
658	# Fin del archivo ajax.py
659	
660	#En el archivo urls.py anexar las siguientes lineas
661	from .ajax import *
662	
663	urlpatterns = [
664	    ....
665	url(r'^listar-users/$', ListUsersAjaxView.as_view(),
666	        name="listar_users"),
667	\end{verbatim}
668	\end{frame}
669	
670	\begin{frame}[fragile]
671	\begin{verbatim}
672	Se anexa el siguiente código el archivo index.html
673	En la parte visual:
674	
675	<div id="datatable"></div>
676	
677	Y en la parte de código javascript
678	
679	<script type="text/javascript">
680	$(document).ready(function() {
681	   $('#datatable').dataTable({
682	        "processing": true,
683	        "serverSide": true,
684	        "ajax": ,
685	        language: {url: JSON_DATA}
686	        });
687	    $('#datatable')
688	        .removeClass('display')
689	        .addClass('table table-striped table-bordered');
690	});
691	</script>
692	\end{verbatim}
693	\end{frame}
694	
695	\subsection{Anexar registros}
696	\begin{frame}[fragile]
697	\begin{verbatim}
698	Creamos el método en el archivo views.py
699	
700	def AnexarRegistro(request):
701	    if request.method == 'POST':
702	       
703	        vusername = request.POST['username']
704	        u = User(username = vusername)
705	        u.save()
706	        message = _("El usuario fue creado")
707	        template = loader.get_template('personal/profile.html')
708	    context = {'message':message}
709	    return HttpResponse(template.render(context, request))
710	       
711	\end{verbatim}
712	\end{frame}
713	
714	\subsection{Modificar y eliminar registros}
715	\begin{frame}[fragile]
716	\begin{verbatim}
717	@login_required
718	def change_profile(request):
719	    """ Cambiar perfil de usuario
720	    """
721	    if request.method == 'POST':
722	        user_id = request.POST['user_id']
723	        personal = User.objects.get(user_id=request.user.id)               
724	        personal.first_name = request.POST['fname']
725	        personal.last_name = request.POST['lname']
726	        personal.email = request.POST['email']
727	        personal.is_active = request.POST.get('is_active')
728	        personal.user.save()
729	       
730	        message = _("The profile change was made")
731	        context = {'personal':personal,'message':message,}
732	    template = loader.get_template('personal/profile.html')
733	    return HttpResponse(template.render(context, request))
734	
735	\end{verbatim}
736	\end{frame}
737	
738	\begin{frame}[fragile]
739	\begin{verbatim}
740	@login_required
741	def delete_personal(request):
742	    """ Delete users
743	    """
744	    id = json.loads(request.POST['seleccionados'])
745	    print id
746	 
747	    for id_values in id:
748	        if id_values["id"] != "0":
749	                   
750	            u = User.objects.get(pk=int(id_values["id"]))
751	            u.delete()   
752	           
753	    return redirect('personal')
754	\end{verbatim}
755	\end{frame}
756	
757	\subsection{View's}
758	\begin{frame}[fragile]
759	\begin{verbatim}
760	####  ListView
761	
762	# En el archivo views.py
763	
764	from django.views.generic.list import ListView
765	from django.utils import timezone
766	
767	from articles.models import Article
768	
769	class ArticleListView(ListView):
770	
771	    model = Article
772	
773	    def get_context_data(self, **kwargs):
774	        context = super(ArticleListView,
775	                        self).get_context_data(**kwargs)
776	        context['now'] = timezone.now()
777	        return context
778	
779	\end{verbatim}
780	\end{frame}
781	
782	\begin{frame}[fragile]
783	\begin{verbatim}
784	
785	# En el archivo urls.py
786	from django.conf.urls import url
787	
788	from article.views import ArticleListView
789	
790	urlpatterns = [
791	 url(r'^$',ArticleListView.as_view(),name='article-list'),
792	]
793	
794	# El template
795	<h1>Articles</h1>
796	<ul>
797	
798	    <li>
799	    {{ article.pub_date|date }} - {{ article.headline }}
800	    </li>
801	
802	    <li>No articles yet.</li>
803	
804	</ul>
805	\end{verbatim}
806	\end{frame}
807	
808	\begin{frame}[fragile]
809	\begin{verbatim}
810	#### DetailView
811	
812	# En el archivo views.py
813	
814	from django.views.generic.detail import DetailView
815	from django.utils import timezone
816	
817	from articles.models import Article
818	
819	class ArticleDetailView(DetailView):
820	
821	    model = Article
822	
823	    def get_context_data(self, **kwargs):
824	        context = super(ArticleDetailView, self).
825	                  get_context_data(**kwargs)
826	        context['now'] = timezone.now()
827	        return context
828	       
829	\end{verbatim}
830	\end{frame}
831	
832	\begin{frame}[fragile]
833	\begin{verbatim}
834	
835	# En el archivo urls.py
836	
837	from django.conf.urls import url
838	
839	from article.views import ArticleDetailView
840	
841	urlpatterns = [
842	 url(r'^(?P<slug>[-\w]+)/$',ArticleDetailView.as_view(),
843	                            name='article-detail'),
844	]
845	
846	# El template
847	
848	<h1>{{ object.headline }}</h1>
849	<p>{{ object.content }}</p>
850	<p>Reporter: {{ object.reporter }}</p>
851	<p>Published: {{ object.pub_date|date }}</p>
852	<p>Date: {{ now|date }}</p>
853	\end{verbatim}
854	\end{frame}
855	
856	\begin{frame}[fragile]
857	\begin{verbatim}
858	#### CreateView
859	
860	from django.views.generic.edit import CreateView
861	from myapp.models import Author
862	
863	class AuthorCreate(CreateView):
864	    model = Author
865	    fields = ['name']
866	
867	# En el template
868	
869	<form action="" method="post">
870	    {{ form.as_p }}
871	    <input type="submit" value="Save" />
872	</form>
873	
874	\end{verbatim}
875	\end{frame}
876	
877	\begin{frame}[fragile]
878	\begin{verbatim}
879	#### UpdateView
880	
881	from django.views.generic.edit import UpdateView
882	from myapp.models import Author
883	
884	class AuthorUpdate(UpdateView):
885	    model = Author
886	    fields = ['name']
887	    template_name_suffix = '_update_form'
888	   
889	# En el template
890	
891	<form action="" method="post">
892	    {{ form.as_p }}
893	    <input type="submit" value="Update" />
894	</form>
895	
896	\end{verbatim}
897	\end{frame}
898	
899	\begin{frame}[fragile]
900	\begin{verbatim}
901	#### DeleteView
902	
903	from django.views.generic.edit import DeleteView
904	from django.urls import reverse_lazy
905	from myapp.models import Author
906	
907	class AuthorDelete(DeleteView):
908	    model = Author
909	    success_url = reverse_lazy('author-list')
910	   
911	# En el template
912	
913	<form action="" method="post">
914	    <p>Are you sure you want to delete "{{ object }}"?</p>
915	    <input type="submit" value="Confirm" />
916	</form>
917	
918	\end{verbatim}
919	\end{frame}
920	
921	\subsection{Crear Login}
922	\begin{frame}[fragile]
923	\begin{verbatim}
924	Creamos la aplicacion user desde la consola:
925	
926	python manage.py startapp user
927	
928	Editamos el archivo user/views.py:
929	
930	# -*- coding: utf-8 -*-
931	
932	from django.shortcuts import render
933	from django.contrib import messages
934	from django.contrib.auth import (
935	    authenticate, logout, login
936	)
937	from django.contrib.auth.models import (
938	    Group, Permission, User
939	)
940	
941	class LoginView(FormView):
942	    form_class = FormularioLogin
943	    template_name = 'users.login.html'
944	    success_url = '/inicio/'
945	\end{verbatim}
946	\end{frame}
947	
948	\begin{frame}[fragile]
949	\begin{verbatim}
950	def form_valid(self, form):
951	       
952	    usuario = form.cleaned_data['usuario']
953	    contrasena = form.cleaned_data['contrasena']
954	    usuario = authenticate(username=usuario,password=contrasena)
955	       
956	    if usuario is not None:
957	        login(self.request, usuario)
958	        messages.info(self.request,'Bienvenido %s has ingresado\
959	                                    Sistema con el usuario %s \
960	                                    ' % (usuario.first_name,
961	                                         usuario.username))
962	    else:
963	        self.success_url = reverse_lazy('users:login')
964	        messages.warning(self.request,'Verifique su nombre y \
965	                                       contraseña\
966	                                      y vuelve a intertar')
967	
968	    return super(LoginView, self).form_valid(form)
969	 
970	\end{verbatim}
971	\end{frame}
972	
973	\subsection{Registro de Usuarios}
974	\begin{frame}[fragile]
975	\begin{verbatim}
976	#### Crear Permisos desde el shell ejecutar
977	
978	from myapp.models import BlogPost
979	from django.contrib.auth.models import Permission
980	from django.contrib.contenttypes.models import ContentType
981	
982	content_type = ContentType.objects.get_for_model(BlogPost)
983	permission = Permission.objects.create(
984	    codename='can_publish',
985	    name='Can Publish Posts',
986	    content_type=content_type,
987	)
988	
989	\end{verbatim}
990	\end{frame}
991	
992	\begin{frame}[fragile]
993	\begin{verbatim}
994	
995	from django.contrib.auth.models import Permission, User
996	from django.contrib.contenttypes.models import ContentType
997	from django.shortcuts import get_object_or_404
998	
999	from myapp.models import BlogPost
1000	
1001	def user_gains_perms(request, user_id):
1002	    user = get_object_or_404(User, pk=user_id)
1003	    # any permission check will
1004	    #cache the current set of permissions
1005	    user.has_perm('myapp.change_blogpost')
1006	
1007	    content_type=ContentType.objects.get_for_model(BlogPost)
1008	    permission = Permission.objects.get(
1009	        codename='change_blogpost',
1010	        content_type=content_type,
1011	    )
1012	    user.user_permissions.add(permission)
1013	
1014	\end{verbatim}
1015	\end{frame}
1016	
1017	\subsection{Login required}
1018	\begin{frame}[fragile]
1019	\begin{verbatim}
1020	
1021	from django.contrib.auth.decorators import
1022	  login_required, permission_required
1023	from django.views.generic import TemplateView
1024	
1025	from .views import VoteView
1026	
1027	urlpatterns = [
1028	    url(r'^about/$', login_required(
1029	           TemplateView.as_view(template_name="secret.html"))),
1030	    url(r'^vote/$', permission_required(
1031	           'polls.can_vote')(VoteView.as_view())),
1032	]
1033	
1034	\end{verbatim}
1035	\end{frame}
1036	
1037	\subsection{Recuperar contraseña por correo}
1038	\begin{frame}[fragile]
1039	\begin{verbatim}
1040	def newpassword(request):
1041	    puser = request.POST['puser']
1042	    try:
1043	        user = User.objects.get(username=puser)
1044	        randompass = ''.join([choice(
1045	        '1234567890qwertyuiopasdfghjklzxcvbnm')
1046	        for i in range(10)])
1047	        print randompass
1048	        subject = _('System: New Password')
1049	        message = _('Your password is reset, new password: ')
1050	                    + randompass
1051	        user.email_user("subject","message")
1052	        user.set_password(randompass)
1053	        user.save()
1054	    except:
1055	        print "error send mail"
1056	        mensaje = _("User not found")
1057	    return redirect('/authenticate/signin')
1058	 
1059	\end{verbatim}
1060	\end{frame}
1061	
1062	\subsection{Conceptos avanzados}
1063	\begin{frame}[fragile]
1064	\begin{verbatim}
1065	Otros conceptos
1066	
1067	* Internacionalización
1068	* Zonas Horarias
1069	* Servidor Web
1070	* Geodjango
1071	* Django Rest
1072	
1073	 Entre otros...
1074	
1075	\end{verbatim}
1076	\end{frame}
1077	
1078	\begin{frame}[plain]
1079	
1080	 
1081	      \begin{center}
1082	
1083	        \font\endfont = cmss10 at 15.40mm
1084	        \color{Brown}
1085	        \endfont 
1086	        \baselineskip 20.0mm
1087	
1088	        CENDITEL
1089	
1090	      \end{center}   
1091	
1092	   
1093	\end{frame}
1094	
1095	
1096	\end{document}
>>>>>>> 392f76c82d86075e983cee03fa4b240023981610
